% !TEX TS-program = xelatex
% !TEX encoding = UTF-8 Unicode
% !Mode:: "TeX:UTF-8"

\documentclass{resume}
\usepackage{zh_CN-Adobefonts_external} % Simplified Chinese Support using external fonts (./fonts/zh_CN-Adobe/)
%\usepackage{zh_CN-Adobefonts_internal} % Simplified Chinese Support using system fonts
\usepackage{linespacing_fix} % disable extra space before next section
\usepackage{cite}

\begin{document}
\pagenumbering{gobble} % suppress displaying page number

\name{罗剑杰}

\basicInfo{
  \email{luojj26@mail2.sysu.edu.cn} \textperiodcentered\ 
  \phone{(+86) 137-1932-6800} \textperiodcentered\ 
  \github[jianjieluo]{https://github.com/jianjieluo}}
 
\section{\faGraduationCap\  教育背景}
\datedsubsection{\textbf{中山大学}, 广州, 广东}{2019 -- 至今}
\textit{在读博士研究生}\ 计算机科学与技术, 预计 2024 年 6 月毕业
\datedsubsection{\textbf{中山大学}, 广州, 广东}{2015 -- 2019}
\textit{学士}\ 软件工程

\section{\faEye\ 研究兴趣}
\begin{itemize}
\item 视觉-语言多模态技术
\item 大规模预训练技术
\end{itemize}

% Reference Test
\section{\faFileText\  论文发表}

\begin{itemize}

 \item \textbf{CoCo-BERT: Improving Video-Language Pre-training with Contrastive Cross-modal Matching and Denoising.} \textbf{Jianjie Luo}, Yehao Li, Yingwei Pan, Ting Yao, Hongyang Chao, Tao Mei. \textit{In ACM Multimedia, 2021}.
 
 \item \textbf{Auto-captions on GIF: A Large-scale Video-sentence Dataset for Vision-language Pre-training.} Yingwei Pan, Yehao Li, \textbf{Jianjie Luo}, Jun Xu, Ting Yao, Tao Mei. \textit{In Arxiv, 2020}.
 
\end{itemize}


\section{\faUsers\ 实习/项目经历}
\datedsubsection{\textbf{京东AI研究院}, 北京}{2018年7月 -- 至今}
\role{研究型实习生}

\begin{itemize}
  \item 在CCF-A国际学术会议 ACM MM2021 上发表论文: CoCo-BERT: Improving Video-Language Pre-training with Contrastive Cross-modal Matching and Denoising. 本人为第一作者。
  \begin{itemize}
    \item 现有视频-文本预训练技术中预训练和微调过程存在输入数据格式不一致的问题,本文在预训练阶段引入不带掩码噪声的输入数据,通过对比学习的方式去减缓上述问题,从而提升预训练模型的性能。
  \end{itemize}
  \item 参与大规模视频-文本跨模态预训练数据集ACTION数据集的构建工作。
  \begin{itemize}
    \item 该数据集包含224,989个视频-文本对。
    \item 该数据集被连续两届 ACM MM2020 和 2021 举办的跨模态预训练比赛所使用。
  \end{itemize}
  \item 参与跨模态分析的多功能和高性能的代码库X-modaler的开发工作。
  \begin{itemize}
    \item 该代码库是业界首个模块化、标准化的跨模态视觉分析代码库。
    \item 该代码库获得了ACM MM2021会议上的最佳开源奖项。
  \end{itemize}
\end{itemize}

\section{\faHeartO\ 获奖情况}
\datedline{\textit{优秀毕业生}, 中山大学}{2019 年 06 月}
\datedline{\textit{国家奖学金}, 中山大学}{2018 年 10 月}
\datedline{\textit{Finalist}, 美国大学生数学建模大赛}{2018 年 04 月}

\section{\faCogs\ IT 技能}
% increase linespacing [parsep=0.5ex]
\begin{itemize}[parsep=0.5ex]
  \item 编程语言: Python, C/C++, MATLAB, LATEX, Markdown, HTML/CSS/JS
  \item 深度学习工具: Caffe, Pytorch
\end{itemize}

% \section{\faInfo\ 其他}
% % increase linespacing [parsep=0.5ex]
% \begin{itemize}[parsep=0.5ex]
%   \item 语言: 英语 - 熟练(六级: 554)
% \end{itemize}

%% Reference
%\newpage
%\bibliographystyle{IEEETran}
%\bibliography{mycite}
\end{document}
